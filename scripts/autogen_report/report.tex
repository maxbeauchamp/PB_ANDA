\documentclass[fleqn]{report}

\renewcommand{\thesection}{\arabic{section}}
\usepackage[utf8]{inputenc}
\usepackage{geometry}
\usepackage{relsize}
\geometry{
 a4paper,total={170mm,257mm},
 left=30mm,right=30mm,top=30mm,bottom=30mm
}
\usepackage{graphicx}
\usepackage{color}
\usepackage{ragged2e}
\usepackage{multicol}
\usepackage[justification=centering]{caption}
\usepackage{subfig}
\usepackage{supertabular}
\usepackage{float}
\PassOptionsToPackage{hyphens}{url}\usepackage{hyperref}
%\newcommand{\justify}{\rightskip0pt \leftskip0pt}
\AtBeginDocument{
 \renewcommand{\figurename}{\small{Figure}}
 \renewcommand{\tablename}{\small{Table }}
}
\renewcommand{\footnoterule}{}

\justifying

\let\oldenumerate=\enumerate
\renewenvironment{enumerate}{\oldenumerate\justifying}{\endlist}

\setlength{\unitlength}{1pt}

\renewcommand{\contentsname}{Table des matières}

\title{Rapport du Lot 1 ``Consolidation des algorithmes basés données et intelligence artificielle pour l'océanographie" \newline \smaller{} \textit{de la Convention 190787/00 ``Méthodes de data science et intelligence artificielle pour la reconstruction des champs altimétrique DUACS}}

\author{Maxime Beauchamp$^{1}$\footnote{Correspondance. Tel.: +33 (0) 6 74 32 47 01.\newline\textit{E-mail address:} maxime.beauchamp76@gmail.com (M. Beauchamp)}, Ronan Fablet$^{1}$ \\ 
{\small 1. IMT Atlantique (anciennement IMT/Télécom Bretagne)} \\  	
{\small UMR 6285 LabSTICC, TOMS (Statistical Signal Processing and Remote Sensing)} \\
{\small Technopole Brest-Iroise - CS 83818} \\
{\small 29238 Brest Cedex 3 (France)} }

\date{Janvier 2019}

\begin{document}

{
\makeatletter
\addtocounter{footnote}{1} % to get dagger instead of star
\renewcommand\thefootnote{\@fnsymbol\c@footnote}%
\makeatother
\maketitle
}

\maketitle

\begin{small}
\tableofcontents
\end{small}


\section{OSSE sans erreurs d'observations}

\subsection{NRMSE et agrégation des données nadir}

\begin{figure}[H]
  \centering
  \includegraphics[width=12cm]{/home3/scratch/mbeaucha/compare_AnDA_nadlag_mod/TS_AnDA_nadir_nadlag.png}
  \caption{NRMSE journalière à partir des données nadir en fonction du pas d'agrégation D$\pm k$, $k=0,\cdots,5$}
\end{figure} 

\begin{figure}[H]
  \centering
  \includegraphics[width=12cm]{/home3/scratch/mbeaucha/compare_AnDA_nadlag_mod/TS_AnDA_nadirswot_nadlag.png}
  \caption{NRMSE journalière à partir des données fusionnées nadir/swot en fonction du pas d'agrégation D$\pm k$, $k=0,\cdots,5$}
\end{figure} 

\subsection{NRMSE et couverture spatiale des données journalières}

\begin{figure}[H]
  \centering
  \includegraphics[width=12cm]{/home3/scratch/mbeaucha/scores_AnDA_nadlag_5_mod/TS_AnDA_nRMSE.png}
  \caption{NRMSE journalière à partir des données nadir, swot et de leur fusion (voir les diagrammes en bâtons pour la couverture spatiale associée), pour les méthodes VE-DINEOF et Post-AnDA, en comparaison de l'OI avec 4 nadir (source CLS).}
\end{figure} 

\subsection{Rendu cartographique et scores (2013--) } 

\subsubsection{Cartes de la SSH et gradient associé}

\begin{figure}[H]
  \hspace{-2cm}\includegraphics[width=18cm]{/home3/scratch/mbeaucha/scores_AnDA_nadlag_0_mod/results_AnDA_maps_2013-09-23.png}
  \caption{Cartographies obtenues par OI, AnDA, Post-AnDA, VE-DINEOF à partir des observations nadir, swot et de leurs fusions, pour la date du }
\end{figure} 

\begin{figure}[H]
  \hspace{-2cm}\includegraphics[width=18cm]{/home3/scratch/mbeaucha/scores_AnDA_nadlag_0_mod/results_AnDA_grads_2013-09-23.png}
  \caption{Cartographies obtenues par OI, AnDA, Post-AnDA, VE-DINEOF à partir des observations nadir, swot et de leurs fusions, pour la date du }
\end{figure}


\subsubsection{Scores}

\begin{figure}[H]
  \centering
  \subfloat[Diagramme de Taylor]{
  \includegraphics[width=15cm]{/home3/scratch/mbeaucha/scores_AnDA_nadlag_0_mod/Taylor_diagram_maps_2013-09-27.png}
  }
  \hfill
  \subfloat[Spectre de puissance]{
  \includegraphics[width=12cm]{/home3/scratch/mbeaucha/scores_AnDA_nadlag_0_mod/results_AnDA_RAPS_2013-09-27.png}
  }
  \caption{Diagramme de Taylor et Spectre de puissance (moyenné radialement) pour la date du }
\end{figure} 

\section{OSSE avec erreurs d'observations}


\end{document}
